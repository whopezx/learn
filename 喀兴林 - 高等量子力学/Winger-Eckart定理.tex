\documentclass{article}

\usepackage[UTF8]{ctex}
\usepackage{amsmath}  % 公式
\usepackage{geometry}  % 设置文档页面格式

\geometry{a4paper,scale=0.9}
\begin{document}
\section*{Winger-Eckart定理}
首先给出这个定理的内容,然后介绍一些前置知识,最后给出证明。

这个定理实际上就是把k阶不可约(球谐)张量算符的矩阵元用角动量基表示出来,即
\begin{equation}
    \langle \tau j'm'|T_{q}^{(k)}|\tau jm\rangle 
    = S_{mqj'm'}^{jk} \frac{1}{\sqrt{2j'+1}}\langle\tau' j'||T^{(k)}||\tau j\rangle
    = (-1)^{j'-m'}
    \begin{pmatrix}
        j' & k & j\\
        -m' & q & m
    \end{pmatrix}
    \langle\tau' j'||T^{(k)}||\tau j\rangle
\end{equation}
这个定理将只与对称性有关的部分分离出来体现在Clebsch-Gorden系数中,
与相互作用有关的部分体现在约化矩阵元中。


\section*{Clebsch-Gordan系数}
CG系数实际上就是一个酉矩阵,将两个子空间直积的基矢$|j_1 m_1\rangle\otimes |j_2 m_2\rangle$
变换为一个大空间的基矢$|j_1 j_2 jm\rangle$。因为这两个子空间的直积就是大空间
所以这两个基矢实际上是同一个空间中的基矢,只是他们之间差一个酉变换,而CG系数就是这个酉变换的矩阵形式。

在给定$j_1$和$j_2$的前提下,即$j_1$和$j_2$是固定的,并且可以取任意轨道角动量$L$或者$S$。
由于$|j_1 m_1 j_2 m_2\rangle = |j_1 m_1\rangle\otimes |j_2 m_2\rangle$是基矢,
则必然有完备关系
\begin{equation}
    \sum_{m_1}\sum_{m_2}|j_1 m_1 j_2 m_2\rangle\langle j_1 m_1 j_2 m_2| = 1
\end{equation}
上式没有对$j_1$和$j_2$求和是因为这两个量是给定的,根据完备关系可以得到
\begin{equation}
    \begin{aligned}
        |j_1 j_2 jm\rangle
        &= |j_1 j_2 jm\rangle\\
        &= \sum_{m_1 m_2}|j_1 m_1 j_2 m_2\rangle\langle j_1 m_1 j_2 m_2|j_1 j_2 jm\rangle\\
        &= \sum_{m_1 m_2}|j_1 m_1 j_2 m_2\rangle S_{m_1 m_2 jm}^{j_1 j_2}
    \end{aligned}
\end{equation}
其中$S_{m_1 m_2 jm}^{jm}=\langle j_1 m_1 j_2 m_2|j_1 j_1 jm\rangle$就是Clebsch-Gorden系数,
接下来通过推导给出具体形式。


\section*{3j符号}





\end{document}